\documentclass[12pt]{article}
\usepackage{graphicx}
\usepackage{paralist}
\usepackage{listings}
\usepackage{hyperref}
\lstset{breaklines = true}
\hypersetup{colorlinks=true,
    linkcolor=blue,
    citecolor=blue,
    filecolor=blue,
    urlcolor=blue,
    unicode=false}



\oddsidemargin 0mm
\evensidemargin 0mm
\textwidth 160mm
\textheight 200mm
\renewcommand\baselinestretch{1.0}
\pagestyle {plain}
\pagenumbering{arabic}
\newcounter{stepnum}
\usepackage{fancyhdr}
\usepackage{fancyhdr}
\fancyhead[L]{\today\ }
\fancyhead[C]{SE 3XA3 Problem Statement}
\fancyhead[R]{Group 2: Genzter}
\pagestyle{fancy}


\title{TimeTable: Problem Statement: Revision 1}
\author{Group 2 \\
		\\ Binu, Amit - binua
		\\ Bengezi, Mohamed - bengezim
		\\ Samarasinghe, Sachin - samarya
		\\ \today\
		\\Professor: Dr. Bokhari
		\\ Lab: L01}
\date{}

\begin {document}
\maketitle
\newpage
\tableofcontents

\newpage

\section{Revision History}
\begin{table}[h]
\begin{center}
\begin{tabular}{ | c | c | c | c | }
\hline
 Date & Version & Description & Author \\ 
\hline
 20/SEP/17 & 0.0 & Created Problem Statement & Mohamed Bengezi \\  
\hline
 21/NOV/17 & 1.0 & Updated Problem Statement & Mohamed Bengezi \\
\hline
 & & & \\
\hline 
 & & & \\ 
\hline 
\end{tabular}
\end{center}
\caption{Revision History}
\end{table}

**Please note that all text in  \textcolor{blue}{blue} has been added/modified since the previous version

\newpage
\section{Problem Statement}
Genzter’s vision is to provide students with a seamless, easy-to-use method of selecting their courses for the upcoming year.\\
 
Picking a schedule for your upcoming school year can be a complicated, time-consuming issue for every student. Finding the perfect balance so that you get all the necessary cores, labs, and tutorials while having no conflicts is no easy task. This is a challenging situation that many students find themselves in. \textcolor{blue}{It is very tedious to continually use trial-and-error to see which combination of cores, labs, and tutorials will not conflict. It would be much easier to have a schedule made for you, where you only need to specify which courses you would like to take. There would be no need to select specific cores, labs, or tutorials.}\\

The Timetable Generator eases the stress of course selection by generating a set of schedules based on the courses the student has selected. There is no need for core, lab, or tutorial selection. \textcolor{blue}{ The user only needs to enter the course code of the desired class(es), and the timetable generator will add all required core's, labs, and tutorials. The user then chooses the most appealing schedule from the set.} \\

The Timetable Generator has the potential to help thousands of students avoid the stress of course selection. \textcolor{blue}{The Timetable Generator will be developed using standard web development languages and frameworks}. It will be available as a web domain, so that any student can access it easily. It also has the potential to be applied to more than just McMaster, or even just universities. It could also be applied to various businesses that revolve around schedules, or are appointment based.  \\

For now, stakeholders include full-time or part-time students at McMaster University, or anyone else that requires a timetable for any McMaster University courses.


    

\bibliographystyle{acm}
\bibliography{ref.bib}

\end{document}  

