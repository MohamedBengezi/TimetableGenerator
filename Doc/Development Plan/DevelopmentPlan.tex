\documentclass[12pt]{article}

\usepackage{graphicx}
\usepackage{paralist}
\usepackage{amsfonts}
\usepackage{listings}
\usepackage{hyperref}

\hypersetup{colorlinks=true,
    linkcolor=blue,
    citecolor=blue,
    filecolor=blue,
    urlcolor=blue,
    unicode=false}

\oddsidemargin 0mm
\evensidemargin 0mm
\textwidth 160mm
\textheight 200mm

\pagestyle {plain}
\pagenumbering{arabic}

\newcounter{stepnum}

\title{Software Development Plan: Revision 0}
\author{Group 2 - Genzter \\
		\\ Binu, Amit - binua - ENTER STUDENT NUMBER
		\\ Bengezi, Mohamed - bengezim - 400021279
		\\ Samarasinghe, Sachin - samarya - ENTER STUDENT NUMBER
		\\ September 29, 2017
		\\Professor: Dr. Bokhari
		\\ Lab: L01}

\begin {document}
\maketitle

\newpage

{\centering
  \tableofcontents\par
}
\addtocontents{toc}{\protect\vspace{2.5em}}

\newpage
\section{Revision History}
\begin{table}[h]
\begin{center}
\begin{tabular}{ | c | c | c | c | }
\hline
 Date & Version & Description & Author \\ 
\hline
 29/SEP/17 & 0.0 & Created development plan & Mohamed Bengezi \\  
\hline
  & & & \\
\hline
 & & & \\
\hline 
 & & & \\ 
\hline 
\end{tabular}
\end{center}
\caption{Revision History}
\end{table}

\newpage
\section{Introduction}
\subsection{Purpose}
The purpose of the Software Development Plan is to gather all information necessary to control the project. It describes the approach to the development of the software and is the top-level plan generated and used by managers to direct the development effort.

The following people use the Software Development Plan:

·         The project manager uses it to plan the project schedule and resource needs, and to track progress against the schedule.

·         Project team members use it to understand what they need to do, when they need to do it, and what other activities they are dependent upon.

\newpage

\section{Team Meeting Plan}
\subsection{Details of Meetings}
Meetings will occur 3 times a week. Two of these meetings will take place during the lab section. The third meeting will take place on Wednesday's, at Thode library, after classes. During each meeting, members will rotate filling the position of Meeting Chair. the Meeting Chair will be in charge of writing the agenda for that meeting, which is discussed below.

\subsection{Meeting Agendas}
As stated above, agendas will be completed by the meeting chair. Meeting agendas will be included in the Meeting Minutes, which will be completed during meeting. This is the \href{run:./docs/MeetingMinutesTemplates.pdf}{Meeting Minutes Template} we will be using. We will also include a summary of decisions made during the meeting, as well as who needs to do what next. The meetings will follow Harvard guidelines, which were discussed in class.

\section{Team Communication Plan}
The team will primarily use a Facebook group chat to communicate, but will also exchange emails as well as phone numbers.
The group chat will be used if any issues come up outside of meetings, or if any group member needs assistance with something non-urgent. For meetings, the Meeting Chair will use both the chat and cell numbers in order to ensure everyone is on the same page. Emails will be used to send/receive data regarding the project.

\newpage

\section{Team Member Roles}
Team member roles will vary, but the following table can provide a general description of each member's current roles. DECIDE 
WHO DOES WHAT
\begin{table}[h]
\begin{center}
\begin{tabular}{ | c | p{65mm} |  }
\hline
 Person & Role  \\ 
\hline
Mohamed Bengezi & - Documentation Lead \newline - Designer \newline - Developer \newline - Test Cases  \\  
\hline
 Amit Binu & - Development Lead \newline - Designer \newline - Documentation assistance \newline - Test Cases  \\
\hline
 Sachin Samarasinghe & - Information Lead \newline - Developer \newline - Code Review \newline - Test Cases
\\
\hline 
\end{tabular}
\end{center}
\caption{Team Member Roles}
\end{table}

\newpage
\section{Git Workflow Plan}
\end {document}
