\documentclass[12pt]{article}
\usepackage{graphicx}
\usepackage{paralist}
\usepackage{listings}
\usepackage{hyperref}
\lstset{breaklines = true}
\hypersetup{colorlinks=true,
    linkcolor=blue,
    citecolor=blue,
    filecolor=blue,
    urlcolor=blue,
    unicode=false}



\oddsidemargin 0mm
\evensidemargin 0mm
\textwidth 160mm
\textheight 200mm
\renewcommand\baselinestretch{1.0}
\pagestyle {plain}
\pagenumbering{arabic}
\newcounter{stepnum}


\title{TimeTable: Problem Statement Revision 0}
\author{Group 2 \\
		\\ Binu, Amit - binua
		\\ Bengezi, Mohamed - bengezim
		\\ Samarasinghe, Sachin - samarya
		\\ September 25, 2017
		\\Professor: Dr. Bokhari
		\\ Lab: L01}
\date{}

\begin {document}
\maketitle
Genzter’s vision is to provide students with a seamless, easy-to-use method of selecting their courses for the upcoming year.\\
 
Picking a schedule for your upcoming school year can be a complicated, time-consuming issue for every student. Finding the perfect balance so that you get all the necessary cores, labs, and tutorials while having no conflicts is no easy task. This is a challenging situation that many students find themselves in. The Timetable Generator eases the stress of course selection by generating a set of schedules based on the courses the student has selected. There is no need for core, lab, or tutorial selection. The user then chooses the most appealing schedule from the set. \\

The Timetable Generator has the potential to help thousands of students avoid the stress of course selection. The Timetable Generator will be developed using Node.js, as well as HTML, CSS, and JS. It will be available as a web domain, so that any student can access it easily. It also has the potential to be applied to more than just McMaster, or even just universities. It could also be applied to various businesses that revolve around schedules, or are appointment based.  \\

For now, stakeholders include full-time or part-time students at McMaster University, or anyone else that requires a timetable for any McMaster University courses.


    

\bibliographystyle{acm}
\bibliography{ref.bib}

\end{document}  

